\subsection{Syntax}
If you are familiar with the python RawConfigParser\footnote{see \url{http://docs.python.org/lib/module-ConfigParser.html}} you should know already how keys and sections are defined.
Musync has a few quirks when it comes to parsing these, so it might be best to read on, even though most things would seem quite familiar.

Configurations are read from /etc/musync.conf and ~/.musync in that respective order.

\emph{The key that is defined last, is the one used}

Musync overwrites it's configuration in \emph{sections}.
These act as \emph{action} references and tell musync how to behave.
And they are used with the \verb!-c! commandline option.

Sections look like the following:

\begin{verbatim}
[foo]
\end{verbatim}

These are followed by a number of keys, which should look like the following.

\begin{verbatim}
[general] # the general section is always loaded
root: /mnt/hdb1/music
coloring: true
[foo]
root: /mnt/hdc1/music
[bar]
coloring: false
\end{verbatim}

\subsection{Keys}
\paragraph{root}
Root directory where files will be handled into. Defaults to None. 

\paragraph{dir}
This gives you a way to tell which directory the files that are created will have. 

\paragraph{format}
This allows you to format 

\paragraph{lock-file}
The file used for storing lock information in. Defaults to None. 

\paragraph{lock}
Set to true or false, depending on if you wan't files to be locked after an add or fix operation. 

\paragraph{default-config}
This is the default configuration string used, when commandline option --config (or -c) is not specified.
This can handle linear references in your configuration, but will fail if you try to configure it with circular references (which could otherwise result in infinite loops). Defaults to None. 

\paragraph{log}
Where the log-file will be written, containing errors and such. Defaults to /tmp/musync.log 

\paragraph{fix-log}
Where the fix-log will be written, containing names of files that need fixes before they can be used with Musync.
These are absolute path, seperated by newlines \verb!\n!. Defaults to /tmp/musync-fixes.log 

\paragraph{add-with}
The command to run when adding files to the depository. Notice the python vars \verb!\%(source)s! and \verb!%(dest)s!. 

\paragraph{rm-with}
The command to run when removing files to the depository. Notice the python var \verb!\%(target)s!. 

\paragraph{filter-with}
The is the command to clean meta data strings with.
The python variable \verb!\%(field)s! is available to specify different files for different meta fields.

example: \verb!/usr/bin/sed -r -f /etc/musync.sed!

or: \verb!/usr/bin/sed -r -f /etc/musync.%(field)s.sed!

{\scriptsize\verb!%(field)s! will be expanded to either \emph{artist}, \emph{album} or \emph{title}}

\paragraph{hash-with}
The command used to get the hash for a file. Output may be just the hash or the hash with a space postfix containing anything. 

\paragraph{check-hash}
Set to true or false, depending on if you want hash checking enabled or not. 

\paragraph{coloring}
Set to true or false, depending on if you want colored output or not. 

\paragraph{overwrite}
If the file that is to be written already exists - and this option is true (which can be set with -O or --overwrite aswell). All files that already exists will be first removed with rm-with before adding file. 

\paragraph{no-fixme}
Prevents fixme-actions from being taken. This might be necessary if you really want to add broken files to your repository.
