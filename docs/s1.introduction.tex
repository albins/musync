\subsection{What is musync?}
\paragraph{The short explanation:}
Musync sorts your music, using the metadata acquired from the files themselves, which exists in most popular music formats as of today.

\paragraph{The slightly longer:}
Musync is a strict, highly customizable music organizer, if it doesn't behave in a logical way -- it is probably a bug.

Musync will \emph{never} modify your original music files, unless you configure it to do so.
it was created to utilise the tools that already exists in a sane operating system, since they are well tested, and provide the highest performance in the safest possible way.

\subsection{Requirements}

\paragraph{compulsary}
\begin{itemize}
\item Python \footnote{\url{http://python.org}}
\item Mutagen library for python\footnote{\url{http://www.sacredchao.net/quodlibet/wiki/Development/Mutagen}}
\end{itemize}

\paragraph{recommended}
\begin{itemize}
\item *nix Coreutils (mv, cp, ln, etc...) \footnote{these exists for windows aswell}
\item GNU Sed \footnote{GNU: \url{http://www.gnu.org/software/coreutils/}}
\end{itemize}


\subsection{Installation}

run
\begin{verbatim}
> setup.py install
\end{verbatim}

as superuser to install. By default musync will try to put all necessary files in
\begin{verbatim}
/usr/share/musync
\end{verbatim}
and it is up to the user to copy these into the right position.

Musync scans the following directories for configuration files:
\begin{itemize}
\item \verb!{/etc/musync.conf}!
\item \verb!\~{}/.musyncrc}!
\end{itemize}
